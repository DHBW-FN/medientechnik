%! Author = mafri
%! Date = 25.07.2022

% Preamble
\documentclass[11pt]{article}

% Packages
\usepackage{amsmath}

% Document
\begin{document}
\chapter{Einleitung}\label{ch:einleitung}
Wie kann Google den Markt der Suchmaschinen mit über 90 Prozent Marktanteil so stark dominieren und welche Konkurrenz gibt es?
Wäre baldiger Wechsel an der Spitze des Marktes denkbar?
In der folgenden Arbeit soll der Markt der Suchmaschinen und einzeln ausgewählte Vertreter analysiert werden.
    Um die Situation auf dem Markt zu vertehen und die wichtigsten Punkte kennenzulernen, in denen sich eine Suchmaschine beweisen muss,
wird im Folgenden zuerst eine Konkurrenzanalyse durchgeführt, bei der zum einen der "big-player" Google,
aber auch der Konkurrent Bing und eine Start-up Suchmaschine namens Qwant betrachtet werden.
Um die Konkurrenzsituation genau zu erläutern werden hier verschiedene Aspekte der Suchmaschinen untersucht, analysiert und dann einander gegenüber gestellt.
    Fortlaufend wird im zweiten Kapitel herausgearbeitet wie Google schon früh aus der Menge der Suchmaschinen herausstach und den Markt bis heute dominiert.
    Hierbei wird Googels Strategie zu unterschiedlichen Zeitpunkten bewertet und untersucht, wie sie auf die heutige Topposition der Suchmaschine einwirkten.
    Vom "big-player" zu einer bisher ziemlich kleinen, unbekannten Suchmaschine,
wird dann das französische Start-up Qwant betrachtet, wessen Suchmaschine sich trotz eines bisher kleinen Marktanteils großer Beliebtheit erfreut.
    Dabei werden die Gründe für den Erfolg von Qwant, seine Chancen und Vorteile gegenüber der bereits genannten Konkurrenz herausgearbeitet.
    Ein wichtiger Aspekt für Suchmaschinen ist ihre Finanzierung, da diese oft gar nicht so offensichtlich ist.
Folgend werden im vierten Kapitel die verschiedenen Finanzierungskonzepte und ihre Rolle in der Aufstellung am Markt untersucht.
    Um die praktische Erfahrung mit den Suchmaschinen Google, Bing und Qwant herauszuarbeiten werden im Anschluss die Ergebnisse von Usability-Tests mit den drei Anbietern präsentiert.
    Diese sollen die Anwendung der Suchmaschinen im Alltag analysieren und so weitere Aspekte, Vor- und Nachteile einzelner Anbieter aufzeigen.
    Zuletzt wird betrachtet, wie Suchmaschinen, vorallem Google, uns als Nutzer beeinflusst oder schon beeinflusst hat.
Auch soll hier ein Blick in die Zukunft geworfen werden und betrachtet werden, wie Google und Co uns weiter beeinflussen könnten und welche Rolle sie in der Zukunft spielen könnten.

\end{document}
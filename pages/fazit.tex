%! Author = mafri
%! Date = 25.07.2022

% Preamble
\documentclass[11pt]{article}

% Packages
\usepackage{amsmath}

% Document
\begin{document}
\chapter{Fazit}\label{ch:fazit}

    \chapter{Fazit}\label{ch:fazit2}

    Google, der \("\)big-player\("\) unter den Suchmaschinen, der mit über 90 Prozent den Markt dominiert,
    wird das wohl auch noch eine Weile tun, denn dass der Gigant innerhalb kurzer Zeit zurückgedrängt oder gar abgelöst wird ist bei den aktuellen Nutzerzahlen eher unwahrscheinlich.
    Trotzdem gibt es einige Konkurrenten wie Qwant auf dem Markt, die vermutlich nicht zu unterschätzen sind.
    Gerade durch einzelne Aspekte, wie Datenschutz, die schon länger kritisch betrachtet werden bei den Branchengiganten wie Google,
    können Start-ups wie Qwant sich aus der Menge heben und dem Nutzer eine gute Alternative bieten.
    Klar ist, dass es eine ganze Weile, vielleicht eine ganz Generation brauchen wird um einen Giganten wie Google abzulösen,
    gerade weil Google es über Jahre hinweg erfolgreich geschafft hat seine Nutzer an sich zu binden.
    Letztlich hängt es wohl hauptsächlich von der Entwicklung einzelner Suchmaschinen in der Zukunft ab und wie gut diese sich an den sich ständig weiterentwickelnden Technikmarkt anpassen können,
    aber Qwant erfreut sich schon jetzt an steigender Popularität und User sind begeistert von den Aspekten, die ihnen Google nicht bieten kann,
    weshalb weiterer Erfolg von kleineren Suchmaschinen und eine Durchmischung des Marktes nicht ausgeschlossen sind.

\end{document}
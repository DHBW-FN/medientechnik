\phantomsection
\newenvironment{keywords}{
	\begin{flushleft}
	\small	
	\textbf{
		\iflanguage{ngerman}{Schlüsselwörter}{\iflanguage{english}{Keywords}{}}
	}
}{\end{flushleft}}

% Deutsche Zusammenfassung
\begin{abstract}
	Das Internet ist aus dem modernen Leben kaum mehr wegzudenken und damit gilt das Selbe für Suchmaschinen wie
	Google, Bing und Co welche das durchforsten des Internets überhaupt erst ermöglichen.
	Um sich die Gunst der Nutzer zu erkämpfen und die stärkste Position in diesem Markt einzunehmen haben sich Suchmaschinen
	stetig weiter entwickelt um den Konkurrenten stets einen Schritt voraus zu sein.\\\\
	Ziel dieser Arbeit ist es einen kurzen Überblick über die Geschichte von Suchmaschinen und im Speziellen den Werdegang
	von Google zur meist genutzten Suchmaschine der Welt zu geben, sowie einen eher unbekannten Konkurrenten des
	Internetgiganten zu betrachten und einen kleinen Ausblick auf die potentielle Zukunft von Suchmaschinen zu geben.\\\\
	Um dieses Ziel zu erreichen werden in dieser Arbeit die Ergebnisse von einer im Rahmen dieses Projekts durchgeführten
	Konkurrenzanalyse sowie eines Usability-Tests mit Google, Bing und Qwant als betrachteten Zielen dargelegt.
	
\end{abstract}

% Schlüsselwörter Deutsch
\begin{keywords}
	Google, Bing, Qwant, Suchmaschinen, Usability, Konkurrenzanalyse, Usability-Test
\end{keywords}


\selectlanguage{english}
% Englisches Abstract
\begin{abstract}
	To imagine a modern world without the internet is nearly impossible today and therefore the same is true for search
	engines like Google, Bing and others because they make it even possible to traverse the internet.
	In order to attract the most users and to gain the most influential position in this market search engines have
	constantly evolved in order to always be one step ahead of their competitors.\\\\
	The aim of the presented work is to give a brief summary of the history of search engines and in particular how
	Google became the most used search engine in the world, as well as to look at a rather less known competitor of the
	internet giant and give a small glimpse of the potential future of search engines.\\\\
	In order to do so the presented work will showcase the results of a competitor analysis as well as a usability test
	with Google, Bing and Qwant as their targets, that were conducted as part of this project.
\end{abstract}

% Schlüsselwörter Englisch
\begin{keywords}
	Google, Bing, Qwant, Search Engines, Usability, Competitor Analysis, Usability Test
\end{keywords}


\selectlanguage{ngerman}
\newpage
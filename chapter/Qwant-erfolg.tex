\section{Erfolg von Qwant}\label{sec:qwant-erfolg}

Wie bereits angekündigt wird in dieser Arbeit besonderes Augenmerk auf einen von Googles eher neueren Konkurrenten
geworfen.
Dabei handelt es sich um Qwant, einer seit 2013 in Frankreich entwickelten Suchmaschine, die besonders mit ihren
strengen Datenschutz-Richtlinien wirbt.
Bevor es später zu einem detaillierten Vergleich der Seiten des Marktführers Google, dem zweitplatzierten Bing und dem
Hauptaugenmerk dieser Arbeit Qwant kommt sollen hier kurz die Vorzüge Qwants gegenüber seinen Mitbewerbern beleuchtet
und ein kurzer Ausblick darüber gegeben werden weshalb es dennoch noch nicht zum großen Durchbruch gereicht hat.

\subsection{Vorzüge von Qwant}\label{subsec:vorzuge-von-qwant}
Ein Vorteil den Qwant gegenüber seinen Mitbewerbern hat, ist das einzigartige Design der Webseite.
Während es in den Anfangstagen des Internets noch ein Vorteil von Google war die eigene Seite sehr schlicht gehalten zu
haben, da dies durch die begrenzten Ressourcen zu besseren Ladezeiten führte, sind solche Limitierungen heut zutage, sogar
in mobilen Bereichen, nahezu nonexistent.
Daher kann die Seite von Qwant mit ihrer farbenfrohen Gestaltung und den optional anzuzeigenden Nachrichtenfenstern sehr
überzeugen, besonders wenn es um mobile Nutzung geht da stärkere und dunklere Farben bei Nutzung im freien deutlich besser
zu erkennen sind.
Dazu kommt die wachsende Produktpalette, die unter anderem Mobile Apps, einen Mapping-Service und eine speziell für
Kleinkinder entwickelte Suchmaschine enthält, und damit durchaus mit Google mithalten kann.
Speziell die Kindersuchmaschine ist ein großer Pluspunkt für die Seite, da das Alter in dem Kinder zum ersten Mal in
Kontakt mit dem Internet treten in den letzten Jahren immer geringer wird und es solch eine spezielle Suchmaschine
sehr einfach ermöglicht Kleinkindern und Minderjährigen den Zugang zu Alters beschränkten Inhalten des Internets
deutlich zu erschweren.
Ebenfalls fällt positiv die Leichtigkeit mit der die Suchmaschine zum Standard im benutzten Browser gesetzt werden kann auf.
Dies kann, beginnend von der Startseite der Suchmaschine, durch zwei bis drei Klicks deutlich zu erkennender Buttons
erreicht und jederzeit rückgängig gemacht werden.
Der jedoch mit Abstand größte Anreiz Qwant zu nutzen ist die strenge Datenschutzpolitik der Suchmaschine, welche dem
immer größer werdenden Bedürfnis der Nutzer nach Schutz im Internet nachkommt.
Während Google, Bing und andere amerikanische Suchmaschinen kein Geheimnis daraus machen die Daten ihrer Nutzer für
verschiedenste Zwecke auszuwerten und gegebenenfalls an Dritte weiterzugeben macht Qwant es sehr deutlich, dass sämtliche
Nutzerdaten geheim bleiben.
Das ist besonders interessant für europäische Nutzer, da Qwant aufgrund der europäischen Herkunft seine Server auch in Europa
stationiert hat auf die Geheimdienste, anders als in den Vereinigten Staaten nur äußerst mühsam Zugriff haben.
Dazu kommen die, seit der neuen Datenschutzgrundverordnung, überall verwendeten Cookie-Banner, welche durch Qwant
vollständig vermieden werden können.

\subsection{Warum kein Durchbruch?}\label{subsec:warum-kein-durchbruch?}
Wie kann es nun also sein das trotz der umfangreichen Vorteile der Seite ein größerer Erfolg bisher ausblieb.
Zum einen liegt das an den unterschiedlichen Umständen in denen wir uns heute befinden.
Anfang der 2000er, als Google, Bing und andere Suchmaschinengrößen um die Vorherrschaft kämpften, gab es noch keine
etablierten Gewohnheiten und Standards in Sachen Suchmaschinen.
Heute hingegen ist Google zum Beispiel standardmäßig als Suchmaschine in Firefox und auf sämtlichen Android Handys
integriert, während Bing auf jedem Windows-Rechner durch Microsoft Edge verwendet wird.
Solche über Jahre entstandenen Standards und Gewohnheiten sind nicht so einfach zu überwinden, denn wie heißt es so
schön: Der Mensch ist ein Gewohnheitstier.
Ein weiterer Grund ist die erst relativ kurze Lebensspanne der Mobilen Applikation(seit 2018) und des Mapping-Dienstes(seit 2019)
welche in einem mittlerweile überwiegend von Mobilgeräten dominierten Markt sehr nachteilhaft sind.
Der größte Grund für das Ausbleiben eines großen Durchbruchs ist aber wohl die starke Konkurrenz speziell im Hinblick auf
den Hauptvermarktungspunkt den erhöhten Datenschutz.
Denn, auch wenn Google, Bing und andere Suchmaschinenriesen noch nicht mit einem Wechsel zu besserem Datenschutz gewechselt
haben, ist Qwant längst nicht die einzige Suchmaschine die mit diesem Feature wirbt.
Auch DuckDuckGo und Ecosia, um nur zwei zu nennen, und die bereits etwas weiter verbreitet sind als Qwant, werben mit
verbessertem Datenschutz und weniger Cookie-Bannern, was Qwant seines größten Verkaufsarguments beraubt.
\section{Usability Tests}
\subsection{Tagebuchstudie}
\subsection{Card Sorting}
Das Card Sorting ist eine Möglichkeit die Struktur und Menüführung einer Seite zu testen.
Es werden Karten erstellt, die die wichtigsten Inhalte umschreiben.
Die Tester schreiben dann den für sie passenden Navigationsbegriff auf die Karte.
Die so entstandenen Karten werden dann von den Testern nach Inhalt gruppiert und diese Gruppen mit Oberbegriffen versehen.
Eine Person dokumentiert, wobei die Tester sich einig waren und welche Punkte umstritten waren.
Dies kann sowohl in Person als auch online erfolgen.\\\\
Diese Testmethode sollte angewendet werden, wenn die Strukturierung der Seite und des Menüs überarbeitet werden soll.
Sie braucht kleinere Gruppen von Testern hat aber einen mittleren Aufwand, da bei jeder Gruppe ein Moderator, der das Vorgehen dokumentiert, erforderlich ist.

\subsection{Eye-Tracking}
Eyetracking wird meist in Kombination mit anderen Usability Tests durchgeführt wie Labortests oder ähnliches.
Beim Eyetracking wird aufgezeichnet auf welche Bereiche des Bildschirms, also der Seite ein Nutzer schaut und wie lange.
Dadurch entstehen neben den subjektiven Daten, die der Nutzer selbst berichtet weitere objektive Daten.
Diese Daten helfen bei der Auswertung der Tests massiv.\\\\
Diese Testmethode sollte angewendet werden, wenn herauszufinden ist, wo wichtige Elemente der Seite platziert werden sollen, wie zum Beispiel ein Kauf-Button,
oder welche Elemente entfernt werden können, da diese von Nutzern nicht beachtet werden.

\subsection{Onsite Befragung}


\subsection{Usability Test im Labor}